
\documentclass[12pt,DIV11,ngerman,a4paper]{scrreprt}

%%%%%%%%%%%%%%%%%%%%%%%%%%%%%%%%%%%%%%%%%%%%%%%%%%%%%%%%%%%%%%%%%%%%%%%%%%
% Einbinden weiterer Pakete % fuer die deutschen Trennmuster
\usepackage[utf8]{inputenc}
\usepackage{listings}

\usepackage[ngerman]{babel} %Sie Umlaute in den Quellen verwenden wollen
%\inputencoding{babel}
\usepackage{amsmath}   % enthaelt nuetzliche Makros fuer Mathematik
\usepackage{amsthm}    % fuer Saetze, Definitionen, Beweise, etc.
\usepackage{amsfonts}
\usepackage{nicefrac}
\usepackage{relsize}
\usepackage{fp}
\usepackage{tikz}

\usepackage{blindtext}
\usepackage{lipsum} % für filler text

\usepackage{graphicx} % für Bilder


% % % % Geometry
\usepackage{relsize}
\usepackage[left=1.5cm,right=1.5cm,top=1.5cm,bottom=1.5cm,includeheadfoot]{geometry}

% % % Fancyhdr für Header / Footer
% \usepackage{fancyhdr}

%%%%%%%%%%%%%%%%%%%%%%%%%%%%%%%%%%%%%%%%%%%%%%%%%%%%%%%%%%%%%%%%%%%%%%%%%%
% Deklaration eigener Mathematik-Makros
\newcommand{\N}{\ensuremath{\mathbb{N}}}   % natuerliche Zahlen
\newcommand{\Z}{\ensuremath{\mathbb{Z}}}   % ganze Zahlen
\newcommand{\Q}{\ensuremath{\mathbb{Q}}}   % rationale Zahlen
\newcommand{\R}{\ensuremath{\mathbb{R}}}   % reelle Zahlen
\newtheorem{aufgabe}[section]{Aufgabe}
\newtheorem{VL}[chapter]{Vorlesung}
\newtheorem{bsp}[subsection]{Beispiel}
\newtheorem{topic}[section]{Thema}
\newtheorem{bem}[subsection]{Bemerkung}
\newtheorem{definition}[subsection]{Definition}
\newenvironment{beweis}%
{\begin{proof}[Beweis]}
	{\end{proof}}

\usepackage[automark,headsepline]{scrlayer-scrpage} % scrpage mit footer die Seitenzahl
\pagestyle{scrheadings}  
\clearpairofpagestyles
\cfoot[\pagemark]{\pagemark}
\newcommand{\code}[1]{\texttt{#1}}
\newcommand{\codation}[1]{\begin{quotation}\code{#1}
	\end{quotation}}

% % % Titlepage Makros
\newcommand{\decr}[1]{\number\numexpr#1-1\relax}
\newcommand{\ExTitle}[1]{
	\selectlanguage{ngerman}
	\setcounter{chapter}{\decr{#1}}
	\date{\today}
	\subject{WSI für Informatik \small an der Karls-Eberhardt Universität Tübingen}
	\author{\large Benjamin Çoban \small 3526251}
	\title{Machine Learning}
	\subtitle{Übungszusammenfassung}
	\maketitle
}
\begin{document}
%\ExTitle{1}
\begin{VL}19.05.2017
\end{VL}

\begin{topic} Stochastik Wiederholung
\end{topic}

\begin{definition}Satz von Beyes
\end{definition}
\begin{align*}P(X\mid Y) = \frac{P(Y\mid X) \cdot P(X)}{P(Y)}
\end{align*}

Das ist der Satz von Beyes, was der bedingten Wahrscheinlichkeit entspricht. 

\begin{align*}P(X\mid Y) = \frac{P(X\cup Y)}{P(Y)}
\end{align*}

\begin{align*}P(A) = \sum_{j\in I} \underbrace{P(A\mid B_j)P(B_j)}_{P(A\cup B_j)}
\end{align*}

$Y$ entspricht den Daten. Alles davon abhängig, was die Messdaten hergeben. Wie kann man eine Klassifikation durchführen? Das $P(Y)$ steckt in jedem Vergleich zwischen zwei Wahrscheinlichkeiten mit drin, ist deshalb auch wegzulassen. 

\begin{definition}
	Bayesche Entscheidungsregel
\end{definition}
Wie gut ist unser $\omega$?
\begin{align*}
	\mathbb{P}(\omega_i\mid x)\\
	\omega(x) \in \Omega\\
	\omega: X \to \Omega\\
	\int_x \mathbb{P}(\omega(x)\mid x)p(x)~\text{d}x \in [0,1]
\end{align*}
Das Integral berechnet den Anteil, wie oft ich mich tatsächlich richtig entschieden habe. 

\begin{topic}Loss Funktion
\end{topic}
$\lambda(\omega_i\mid \omega_j)$: in Wirklichkeit $\omega_j$, klassifiziert $\omega_i$ beschreibt die \underline{Fehlerdramatik}
\begin{align*}R(\omega_i\mid x) = \sum_{j = 1}^{N} \lambda(\omega_i|\omega_j)\cdot \mathbb{P}(\omega_j\mid x)
\end{align*}
Das berechnet also das \underline{Risiko} meiner Entscheidungen. Gesamtrisiko:
\begin{align*}
\mathcal{R}(\omega) = \int_x R(\omega(x)\mid x)\cdot p(x)~\text{d}x
\end{align*}
Und das möchte ich natürlich minimieren. Komme somit auf das \underline{Baye'sche Gesamtrisiko}

\begin{topic}UE1:Aufgabe 2
\end{topic}

$\sigma$ Standardabweichung, $\mu$ Varianz
\end{document}