\setcounter{chapter}{3}
\chapter{Korrektur}

\begin{aufgabe}
\end{aufgabe}

Monotonie

c) wichtige Annahmen: gibt es eine Verteilung meiner Parameter? Unabhängigkeitsannahme. Annahme der parametrischen Form.
d) Fehler: Bayesfehler: Wenn sich die Varianz überlappt, kann man nie perfekt entscheiden. Falsche parametrische Form, größerer Schätzfehler als minimal möglich.

\begin{aufgabe}
\end{aufgabe}

Gegeben sind Ergebnisse $x_1,...,x_n$ von unabhängigen und identisch verteilten Zufallsvariablen $X_1,...,X_n$. 
Ableiten, Kettenregel, Nullstellen berechnen.

\begin{align*}
\ln(p(x))=
bla+\frac{-(x-\mu)^2}{2\sigma^2}\\
\frac{d}{d\mu} \ln(p(\mu)) = \frac{x-\mu}{\sigma^2}\\
\sum \frac{x_i - \mu}{\sigma^2} = 0\\
\Rightarrow \sum x_i = n\cdot \hat{\mu}
\end{align*}
Erste Aufgabe: Was ist der maxlikelihood Schätzer für $\mu$ normalverteilt
\begin{align*}
\hat{\mu} = \frac{1}{n} \sum x_i\\
\hat{\sigma}^2 = \frac{1}{n} \sum (x_i - \hat{\mu})^2
\end{align*}
$\hat{\mu}$ als maximum, siehe analysis aus der Schule (Ableiten bla). Wollen ja
\begin{align*}
\begin{pmatrix}\mu \\
\sigma ^2
\end{pmatrix} \to p(\mu,\sigma^2)
\end{align*}
zu maximieren. Also: Gradient bilden (Mathe II)
\section{Anmerkungen}
max\_likelihood: bei wenigen Daten mit Vorischt zu genießen. (Würfel 3x würfeln und dann doof schätzen wollen). Bei der Bayesschen Schätzung ist dafür die a priori Wahrscheinichkeit da, um des einzudämmen. Quasi ein Würfelmodell implementiert durch Bayes.