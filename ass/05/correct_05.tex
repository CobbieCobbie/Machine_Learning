\ExTitle{5}
\begin{topic}
	Korrektur
\end{topic}

\begin{aufgabe}
\end{aufgabe}
Sollte im großen und ganzen so passen.

\begin{aufgabe}
\end{aufgabe}
Wir brauchen bei den Wolken eine schräge Gerade, damit man den größten Abstand betrachtet. Die Wolken sind eine Klasse quasi, darum betrachtet man die größte Differenz über \underline{alles}! Ob es gut ist oder nicht für die Klassifizierung, ist einfach nur dem Bild zu entnehmen. Evtl sorgt die PCA dafür, dass bei den vorherig disjunkten Wolken dann eine nichtleere Schnittmenge entsteht und somit ein Fehlerpotenzial für die Klassifizierung. 