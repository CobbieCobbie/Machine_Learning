
\documentclass[12pt,DIV11,ngerman,a4paper]{scrreprt}

%%%%%%%%%%%%%%%%%%%%%%%%%%%%%%%%%%%%%%%%%%%%%%%%%%%%%%%%%%%%%%%%%%%%%%%%%%
% Einbinden weiterer Pakete % fuer die deutschen Trennmuster
\usepackage[utf8]{inputenc}
\usepackage{enumerate}

\usepackage[ngerman]{babel} %Sie Umlaute in den Quellen verwenden wollen
%\inputencoding{babel}
\usepackage{amsmath}   % enthaelt nuetzliche Makros fuer Mathematik
\usepackage{amsthm}    % fuer Saetze, Definitionen, Beweise, etc.
\usepackage{amsfonts}
\usepackage{nicefrac}
\usepackage{relsize}
\usepackage{fp}
\usepackage{tikz}

\usepackage{blindtext}
\usepackage{lipsum} % für filler text

\usepackage{graphicx} % für Bilder


% % % % Geometry
\usepackage{relsize}
\usepackage[left=1.5cm,right=1.5cm,top=1.5cm,bottom=1.5cm,includeheadfoot]{geometry}

% % % Fancyhdr für Header / Footer
% \usepackage{fancyhdr}

%%%%%%%%%%%%%%%%%%%%%%%%%%%%%%%%%%%%%%%%%%%%%%%%%%%%%%%%%%%%%%%%%%%%%%%%%%
% Deklaration eigener Mathematik-Makros
\newcommand{\N}{\ensuremath{\mathbb{N}}}   % natuerliche Zahlen
\newcommand{\Z}{\ensuremath{\mathbb{Z}}}   % ganze Zahlen
\newcommand{\Q}{\ensuremath{\mathbb{Q}}}   % rationale Zahlen
\newcommand{\R}{\ensuremath{\mathbb{R}}}   % reelle Zahlen
\newtheorem{aufgabe}[section]{Aufgabe}
\newenvironment{beweis}%
{\begin{proof}[Beweis]}
	{\end{proof}}


\usepackage[automark,headsepline]{scrlayer-scrpage} % scrpage mit footer die Seitenzahl
\pagestyle{scrheadings}  
\clearpairofpagestyles
\cfoot[\pagemark]{\pagemark}

% % % Titlepage Makros
\newcommand{\decr}[1]{\number\numexpr#1-1\relax}
\newcommand{\ExTitle}[1]{
	\selectlanguage{ngerman}
	\setcounter{chapter}{\decr{#1}}
	\date{\today}
	\subject{WSI für Informatik \small an der Karls-Eberhardt Universität Tübingen}
	\author{\large Lea Bey - Benjamin Çoban - Thomas Stüber}
	\title{Machine Learning}
	\subtitle{Übungsblatt #1}
	\maketitle
	\chapter{}
}
\begin{document}
\ExTitle{4}
\setcounter{chapter}{3}
\chapter{Korrektur}

\begin{aufgabe}
\end{aufgabe}

Monotonie

c) wichtige Annahmen: gibt es eine Verteilung meiner Parameter? Unabhängigkeitsannahme. Annahme der parametrischen Form.
d) Fehler: Basefehler: Wenn sich die Varianz überlappt, kann man nie perfekt entscheiden. Falsche parametrische Form, größerer Schätzfehler als minimal möglich.

\begin{aufgabe}
\end{aufgabe}

Gegeben sind Ergebnisse $x_1,...,x_n$ von unabhängigen und identisch verteilten Zufallsvariablen $X_1,...,X_n$. 
Ableiten, Kettenregel, Nullstellen berechnen.

\begin{align*}
\ln(p(x))=
bla+\frac{-(x-\mu)^2}{2\sigma^2}\\
\frac{d}{d\mu} \ln(p(\mu)) = \frac{x-\mu}{\sigma^2}\\
\sum \frac{x_i - \mu}{\sigma^2} = 0\\
\Rightarrow \sum x_i = n\cdot \hat{\mu}
\end{align*}
Erste Aufgabe: Was ist der maxlikelihood Schätzer für $\mu$ normalverteilt
\begin{align*}
\hat{\mu} = \frac{1}{n} \sum x_i\\
\hat{\sigma}^2 = \frac{1}{n} \sum (x_i - \hat{\mu})^2
\end{align*}
$\hat{\mu}$ als maximum, siehe analysis aus der Schule (Ableiten bla). Wollen ja
\begin{align*}
\begin{pmatrix}\mu \\
\sigma ^2
\end{pmatrix} \to p(\mu,\sigma^2)
\end{align*}
zu maximieren. Also: Gradient bilden (Mathe II)
\end{document}