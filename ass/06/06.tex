\ExTitle{6}
\begin{aufgabe}
\end{aufgabe}
\begin{aufgabe}
\end{aufgabe}
Betrachte das Bayessche Netz mit den verteilten Wahrscheinlichkeiten auf dem Übungsblatt. Zu berechnen sind folgende Wahrscheinlichkeiten:
\begin{enumerate}[a)]
	\item $p(x_1)$
	\begin{align}
		p(x_1) &= \sum_{i,j,k,l} p(x_1,a_i,b_j,c_k,d_l)\\
		&\underbrace{=}_{Graph} \sum_{i,j,k,l} p(x_1\mid a_i,b_j)\cdot p(a_i)\cdot p(b_j) \cdot p(c_k\mid x_1) \cdot p(d_l\mid x_1)\\
	    &= \sum_{i,j} \sum_{k,l} p(x_1\mid a_i,b_j)\cdot p(a_i)\cdot p(b_j) \cdot p(c_k\mid x_1) \cdot p(d_l\mid x_1)\label{a:1}
	\end{align}
	In Zeile \ref{a:1} betrachten wir nun $p(c_k\mid x_1)$ und $p(d_l\mid x_1)$. Dies heißt, dass wir über eine normierte Zeile iterieren, sowohl $p(c_k\mid x_1)$ als auch $p(d_l\mid x_1)$ summieren sich zu 1 auf, wenn wir sie ausklammern. Somit fällt die innere Summe mit den einzelnen Wahrscheinlichkeiten raus. Übrig bleibt also:
	\begin{align}
	= \sum_{i,j} p(x_1\mid a_i,b_j)\cdot p(a_i)\cdot p(b_j)
	\end{align}
	Macht auch Sinn, denn es interessiert an Hand vom Graphen für die Wahrscheinlichkeit von Lachs nicht, ob er o.B.d.A. hell oder breit ist.
	\begin{align}
	&= p(a_i) \cdot \sum_{i,j} p(x_1\mid a_i,b_j) \cdot p(b_j)\\
	&= 0,25 \cdot (0,5 \cdot 0,6 + 0,7\cdot 0,4 + 0,6\cdot 0,6 + 0,8\cdot 0,4 \\
	&+ 0,4\cdot 0,6 + 0,1\cdot 0,4 + 0,2\cdot 0,6 + 0,3\cdot 0,4) = \underline{0,445}
	\end{align}
	\item $p(a_1,b_1,x_1,c_1,d_2)$
	\begin{align*}
	&= p(a_1) \cdot p(b_1) \cdot p(x_1\mid a_1,b_1) \cdot p(c_1\mid x_1) \cdot p(d_2\mid x_1)\\
	&= 0,25 \cdot 0,6 \cdot 0,5 \cdot 0,6 \cdot 0,7 = 0,0315
	\end{align*}
	\item $p(x_1\mid c_2)$\\
	Berechne zuerst $p(c_2)$:
	\begin{align*}
		p(c_2) = \sum_{i,j,k,l} p(a_i,b_j,x_k,c_2,d_e)\\
	\end{align*}
	Da wir in Teilaufgabe 1 bereits die Wahrscheinlichkeit für Lachs - und somit auch für Barsch - berechnet haben, kann ein erheblicher Teil der Rechnung übersprungen werden. Das liegt am Bayesschen Netz: Die Wahrscheinlichkeit eines Knotens ist bedingt durch seine Elternknoten - in unserem Fall X und bereits ausgerechnet.
	\begin{align*}
		&= \sum_{i} p(c_2\mid x_i)  \cdot p(x_i)\\
		&= 0,2\cdot 0,445 + 0,3 \cdot 0,555 = 0,2555 
	\end{align*}
	\begin{align*}
		p(x_1\mid c_2) &= \frac{p(c_2,x_1)}{p(c_2)}\underbrace{=}_{\texttt{Satz von Beyes}} \frac{p(x_1)\cdot p(c_2\mid x_1)}{p(c_2)} = 0,3483
	\end{align*}
	\item $p(a_3\mid x_2,d_1)$
	Laut Tutorium gilt folgende Gleichung:
	\begin{align}
		p(a_3\mid x_2,d_1) = \frac{p(a_3,x_2,d_1)}{p(x_2)\cdot p(d_1\mid x_2)}\label{d:ag}
	\end{align}
	Berechne nun den Zähler aus \ref{d:ag}
	\begin{align*}
	p(a_3,x_2,d_1) &= \sum_{i,j} p(a_3)\cdot p_(b_i) \cdot p(x_2\mid a_3,b_i) \cdot \underbrace{p(c_j\mid x_2)}_{=1}\\
	&= p(a_3) \cdot p(d_1\mid x_2) \cdot \sum_{i} p(b_i) \cdot p(x_2\mid a_3,b_i)\\
	&= 0,25 \cdot 0,6 \cdot (0,6\cdot 0,6 + 0,4 \cdot 0,9) = 0,108
	\end{align*}
	Setzen wir das ganze nun in \ref{d:ag} ein, erhalten wir:
	\begin{align*}
		p(a_3\mid x_2,d_1) = \frac{0,108}{0,555\cdot 0,6} = 0,3243
	\end{align*}
	\item $p(c_1\mid d_2,x_1)$\\
	Das Vorgehen ist gleich wie bei Teilaufgabe d)
	\begin{align*}
		p(x_1,c_1,d_2) &= \sum_{i,j} p(a_i)\cdot p(b_j) \cdot p(x_1\mid a_i,b_j) \cdot p(c_1\mid x_1)\cdot p(d_2\mid x_1)\\
		&=p(c_1\mid x_1)\cdot p(d_2\mid x_1) \cdot \underbrace{\sum_{i,j} p(a_i) \cdot p(b_j) \cdot p(x_1\mid a_i,b_j)}_{=p(x_1)}
	\end{align*}
	Einsetzen in \ref{d:ag}
	\begin{align*}
	p(c_1\mid d_2,x_1) &= \frac{p(c_1\mid x_1)\cdot p(d_2\mid x_1)\cdot p(x_1)}{p(d_2\mid x_1)\cdot p(x_1)}\\
	&= p(c_1\mid x_1) = 0,6
	\end{align*}
	
\end{enumerate}